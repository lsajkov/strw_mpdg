%% AMS-LaTeX Created with the Wolfram Language : www.wolfram.com

\documentclass{article}
\usepackage{amsmath, amssymb, graphics, setspace}

\newcommand{\mathsym}[1]{{}}
\newcommand{\unicode}[1]{{}}

\newcounter{mathematicapage}
\begin{document}

\title{Finding conversion ratio for \(\) to \(\)}
\author{}
\date{}
\maketitle

The aim is to find the factor that will convert the half-light radius of an observation to the FWHM of an idealized symmetric Gaussian.

Define 2d Gaussian using multimodal normal distribution. Assume it is symmetric, with the same variance in both directions. Define it in radial coordinates.

\begin{doublespace}
\noindent\(\pmb{\text{gauss2d}=\text{Simplify}[\text{PDF}[\text{MultinormalDistribution}[\{\{\sigma {}^{\wedge}2,0\},\{0,\sigma {}^{\wedge}2\}\}],\{r
\text{Cos}[\theta ], r \text{Sin}[\theta ]\}], \sigma >0]}\)
\end{doublespace}

\begin{doublespace}
\noindent\(\frac{e^{-\frac{r^2}{2 \sigma ^2}}}{2 \pi  \sigma ^2}\)
\end{doublespace}

Full {``}power{''} contained within the Gaussian (across entire real plane):

\begin{doublespace}
\noindent\(\pmb{\text{Integrate}[\text{gauss2d},\{r,0,\text{Infinity}\},\{\theta ,0,2\text{Pi}\}]}\)
\end{doublespace}

\begin{doublespace}
\noindent\(\frac{\sqrt{\frac{\pi }{2}}}{\sigma }\)
\end{doublespace}

Find radius that encloses half the {``}power{''} in a Gaussian with the same $\sigma $.

\text{Integrate}[\text{gauss2d}, \{r, 0, \text{R50}\} ,\{\theta , 0, 2\text{Pi}\}]

\begin{doublespace}
\noindent\(\frac{\sqrt{\frac{\pi }{2}} \text{Erf}\left[\frac{\text{R50}}{\sqrt{2} \sigma }\right]}{\sigma }\)
\end{doublespace}

\(\)\\
\\
We are therefore looking for the solution to:\\
\\
\(\)\\
Apply \(\) to both sides of the equation. Find a numerical approximation to this value

\begin{doublespace}
\noindent\(\pmb{N\left[\text{InverseErf}\left[\frac{1}{2}\right]\right]}\)
\end{doublespace}

\begin{doublespace}
\noindent\(0.476936\)
\end{doublespace}

This is, approximately,

1/0.476

\begin{doublespace}
\noindent\(2.10084\)
\end{doublespace}

translating to:\\
\\
\(\)\\
\\
or\\
\\
\(\)\\
\\
for the corresponding 1-d Gaussian (the marginal PDF in either \textit{ x }or \textit{ y}).\\
\\
The full width at half maximum for a 1-d Gaussian is given by:\\
\\
\(\)\\
\\
translating to a radius of:\\
\\
\(\)\\
\\
Therefore,\\
\\
\(\)\\
or\\
\(\)

\begin{doublespace}
\noindent\(\pmb{(2.1\ 2.355)/(2\text{Sqrt}[2])}\)
\end{doublespace}

\begin{doublespace}
\noindent\(1.7485\)
\end{doublespace}

Finally, we get a value of:\\
\(\) 

\end{document}
